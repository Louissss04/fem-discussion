\documentclass[12pt,a4paper]{article}
\usepackage{geometry}
\geometry{left=2.5cm,right=2.5cm,top=2.0cm,bottom=2.5cm}
\usepackage[english]{babel}
\usepackage{amsmath,amsthm}
\usepackage{amsfonts}
\usepackage[longend,ruled,linesnumbered]{algorithm2e}
\usepackage{fancyhdr}
\usepackage{ctex}
\usepackage{array}
\usepackage{listings}
\usepackage{color}
\usepackage{graphicx}
\usepackage{amssymb}
\begin{document}
	
	\noindent
	
	\section*{2024.03.13}	
	
	
	考虑两点边值问题
	\begin{equation}
		\left\{
		\begin{array}{l}
			-u'' + u = (\pi^2 +1) \sin{\pi x}, \\
			u(0) = u(1) = 0.
		\end{array}
		\right.
	\end{equation}
	
	\begin{enumerate}
		
		\item 验证变分问题 $$B(u_n, v_n) = (f, v_n),\quad \forall v_n \in V_n$$ 满足Lax-Milgram 定理条件,解$u_n$存在且唯一, 并满足范数估计 $\|u_n\|_V \leq \frac{1}{\alpha} \|f\|_{V^*} $.
		
		\item 用 matlab 实现有限元程序计算该一维边值问题.(取N = 4, 8, 16, 32, 用稀疏矩阵存储)
		
		
	\end{enumerate}
	
	\section*{2024.03.20}
	
	\begin{enumerate}
		\item 设有椭圆方程混合边值问题
		$$
		\begin{cases}-\Delta u+a(\boldsymbol{x}) u(\boldsymbol{x})=f(\boldsymbol{x}), & \boldsymbol{x} \in \Omega \\ \frac{\partial u}{\partial n}+\beta(\boldsymbol{x}) u(\boldsymbol{x})=g(\boldsymbol{x}), & \boldsymbol{x} \in \partial \Omega\end{cases}
		$$    
		
		其中 $\Omega$ 为 $\mathbb{R}^n$ 中具有光滑边界 $\partial \Omega$ 有界单连通区域,$\frac{\partial u}{\partial n}$ 为法向导数, $a(\boldsymbol{x}) \geqslant a_0>0, \beta(\boldsymbol{x}) \geqslant 0$, $f(\boldsymbol{x})$ 均为足够光滑的已知函数.
		\item[(i)] 试建立该边值问题的 Galerkin 变分问题;
		\item[(ii)] 讨论 (i) 中变分问题解的适定性.
		
		\item 对于椭圆型方程的第一边值问题
		$$
		\left\{\begin{array}{l}
			-\partial_1\left(p(\boldsymbol{x}) \partial_1 u\right)-\partial_2\left(p(\boldsymbol{x}) \partial_2 u\right)+q(\boldsymbol{x}) u(\boldsymbol{x})=f(\boldsymbol{x}), \quad \boldsymbol{x} \in \Omega, \\
			\left.u(\boldsymbol{x})\right|_{\Gamma}=g(\boldsymbol{x}),
		\end{array}\right.
		$$   
		
		其中 $\Omega$ 为 $\mathbb{R}^2$ 中具有光滑边界 $\Gamma$ 的有界单连通区域,$p(\boldsymbol{x}) \geqslant p_0>0, q(\boldsymbol{x}) \geqslant 0, f(\boldsymbol{x})$ 足够光滑. 试建立该边值问题的 Galerkin 变分问题, 并研究其适定性.
		
	\end{enumerate}
	
	
	
\end{document}

